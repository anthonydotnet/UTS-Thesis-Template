% Environment & debugging
%\RequirePackage[l2tabu]{nag} % Detecting and warning about obsolete LATEX commands
\usepackage{import}          % Establish input relative to a directory
\usepackage{verbatim}        % Reimplementation of and extensions to LATEX verbatim. Useful for text-blocks and code-blocks
% \usepackage{lineno}         % Line numbers 
%\usepackage{alltt}           % tt font, but obey commands and line endings. LaTeX commands are not disabled in verbatim-like environment
%\usepackage{url}             % Verbatim with URL-sensitive line breaks
%\usepackage{lipsum}	      % Create dummy text
%\usepackage{layouts} 	      % Display various elements of a document's layout

% Date formatting
% From https://tex.stackexchange.com/questions/162353/memoir-class-conflict-with-datetime
%\let\ordinal\relax    % Relax ordinal warning - remove if memoir has issues. 
%\usepackage{datetime} % This causes warning about \ordinal
\usepackage[en-GB]{datetime2}

% Structure & Layout
\usepackage{calc,soul,fourier} % Arithmetic in LATEX commands. Required for Chapters
\usepackage{rotfloat} % Rotate floats. E.g. tables, figures. Used for Chapter number box
\usepackage{parskip} % Layout with zero \parindent, non-zero \parskip - Anthony
%\usepackage{balance} % Balanced two-column mode
%\usepackage{float}   % Improved interface for floating objects
\usepackage{mdframed} % Frames and boxes
\usepackage{fancybox}
\usepackage{pdflscape}


% Fonts & Symbols
\usepackage{newlfont}     % Helpful package for fonts and symbols. Required by memoir.
\usepackage{calligra}     % Handwriting style. Used for dedication
%\usepackage{fouriernc}    % Font with math support: New Century Schoolbook
%\usepackage{frcursive}    % French academic handwriting
%\usepackage{epigraph}     % For typesetting epigraphs
%\usepackage[T1]{fontenc}  % ‘font-encoding-specific’ commands. E.g. custom hyphenation
%\usepackage{textcomp}     % Text Companion fonts. E.g. bullet, copyright, musicalnote

% Colour, Images, & Graphics
\usepackage[table]{xcolor} % Driver-independent color extensions for LATEX and pdfLATEX
\usepackage{tikz}          % For absolute positioning of images. Used on declaration page
%\usepackage{svg}          % For SVG images
%\usepackage[inkscapelatex=false]{svg}

%\usepackage{color}        % Colour control for LATEX documents
%\usepackage{graphicx}     % Enhanced support for graphics
%\usepackage{epsf} - Using graphicx - Anthony
%\usepackage{epsfig} - Using graphicx - Anthony
\usetikzlibrary{shapes.geometric, arrows.meta, positioning, calc}

% Citation, referencing
\usepackage[colorlinks=true,allcolors=black]{hyperref} % Hypertext. E.g. Creates hyperlinks in cross references
%\usepackage{footnote} % Improve on LATEX's footnote handling
%\usepackage[square,numbers,sort&compress]{natbib} % Flexible bibliography support
%\usepackage{memhfixc} % Adjustment for using hyperref in memoir documents. Use on memoir document class after hyperref
\usepackage{bibentry} \nobibliography* % used for inline referencing on the publications page
\usepackage[acronyms,toc,nonumberlist]{glossaries-extra}



% Tables
%\usepackage{booktabs}                  % Publication quality tables in LATEX
%\usepackage{longtable,rotating}	    % Allow tables to flow over page boundaries

%\usepackage{xtab}


%\usepackage{tabularx}
%\usepackage{adjustbox} 
%\usepackage{slashbox}                  % Both column and row headings in a tabular cell
%\usepackage[flushleft]{threeparttable} % Tables with captions and notes all the same width
%\usepackage{multirow}                  % Create tabular cells spanning multiple rows
%\usepackage{array}                     % Extends the options for column formats
%\usepackage{boldline}                  % Heavier lines in tables
\usepackage{tablefootnote}              % Permit footnotes in tables




% Lists
\usepackage{enumitem}
\usepackage{enumerate}
%\usepackage{paralist} % Enumerate and itemize within paragraphs
%\usepackage{mdwlist}  % Miscellaneous list-related commands


% PDF & Encoding
%\usepackage{ifpdf}              % For metadata
%\usepackage{epstopdf}           % Perl script that converts an EPS file to an ‘encapsulated’ PDF file
%\usepackage[applemac]{inputenc} % Accept different input encodings. E.g. non-english characters for mac
%\usepackage{enumerate}	         % Enumerate with redefinable labels. Style for counter.
%\usepackage{microtype}	         % Makes pdf look better.

